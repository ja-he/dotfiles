\documentclass{article}

  % input & font encoding 
    \usepackage[utf8]{inputenc}
    \usepackage[english]{babel}
    % \usepackage[ngerman]{babel}
    % hangul character parsing and display
    \usepackage{kotex}
    \usepackage{tgschola}
    \usepackage[T1]{fontenc}
    % math
    \usepackage{mathtools}
    \usepackage{amsmath}
    \usepackage{amssymb}
    \usepackage{amsthm}
    \usepackage{amsfonts}
    \usepackage{amsbsy}
    \usepackage{latexsym}
    \usepackage{dsfont}

  % tables 
    % better tabular alternative 
    \usepackage{tabulary}
    % pretty tables
    \usepackage{booktabs}
      \renewcommand{\arraystretch}{1.2}
    % no paragraph start indentation 
      \setlength{\parindent}{0em}
    % small gaps between paragraphs 
      \setlength{\parskip}{0.5em}

  % colors 
    \usepackage{xcolor}
      \definecolor{blue_custom} {HTML}{0066b5}
      \definecolor{red_custom}  {HTML}{9e0007}
      \definecolor{green_custom}{HTML}{007527}
      \definecolor{gray_custom} {HTML}{999999}
      \definecolor{todored}{HTML}{ff7c7a}
      \newcommand{\gr}[0]{\color{green_custom}}
      \newcommand{\re}[0]{\color{red_custom}}
      \newcommand{\bl}[0]{\color{blue_custom}}
      \newcommand{\bk}[0]{\color{black}}
      \newcommand{\wh}[0]{\color{white}}

  % moving stuff around...
    \usepackage[left=3cm,right=7cm,top=3cm,bottom=3cm]{geometry}
    \usepackage{stackengine}
    \usepackage{textcomp}

  % TODOs 
    \usepackage[textwidth=4cm,
                textsize=tiny,
                color=todored,
                bordercolor=white]{todonotes}
    % deprecated custom todo command
    % \newcommand{\todo}[1]{\texttt{\re TODO #1\\}}

  % code listings 
    \usepackage{listings}
      \lstset{
        basicstyle=\ttfamily\footnotesize, 
        breaklines=true,
        commentstyle=\color{gray_custom},
        keywordstyle=\color{blue_custom},
        stringstyle=\color{green_custom}, 
        morekeywords={},
        deletekeywords={},
        keepspaces=true,
        frame=none,
        language=Java,
        numbers=left,
        numberstyle=\tiny\color{gray_custom},
        showspaces=false,
        showtabs=false,
        showstringspaces=false,
        xleftmargin=0.05\textwidth,
        xrightmargin=0.05\textwidth, 
        linewidth=\textwidth, 
        tabsize=4,
        columns=fullflexible,
        captionpos=t,
        gobble=0
      }
      \lstdefinestyle{bash}{
        language=Bash,
        morekeywords={sleep,ffmpeg,chmod,grep,rm,mv,cp,vim}
      }
      \lstdefinestyle{terminal}{
        style=bash,
        numbers=none
      }
      \lstdefinestyle{cpp}{
        language=C++
      }
      \lstdefinestyle{haskell}{
        language=haskell
      }
      \lstdefinestyle{latex}{
        language=tex,
        numbers=none
      }

  % theo2 vorlage 
    % blank / cent commands 
    \usepackage[makeroom]{cancel}
      \renewcommand{\b}{\ensuremath{\cancel{b}}}
      \renewcommand{\c}{\ensuremath{\cancel{c}}}
    % tikz für automatengrafen
    \usepackage{tikz}
      \usetikzlibrary{arrows,automata}

  % custom commands 
    % math explanations 
    % \mexpl: regular explanation 
      \newcommand{\mexpl}[1]{\color{gray_custom}\left(\text{#1}\right)}
    % \mexplbig: expansive multi-line explanation 
      \newcommand{\mexplbig}[1]{ \color{gray_custom}{\left(\parbox{4cm}{
              \centering \footnotesize #1 }\right)} }

\author{Jan Hensel}
\title{\textbf{\LaTeX\ Template File}\\
        Example Secondary Title}
\begin{document}
\maketitle

\section{Specific Settings}
  \subsection{Margins}
    The margins are set via the \texttt{geometry}-package via the settings
    \texttt{left}, \texttt{right}, \texttt{top} and \texttt{bottom} such that
    the ultimate inclusion looks as follows: 
    \begin{lstlisting}[style=latex,gobble=6]
      \usepackage[left=3cm,
                  right=7cm,
                  top=3cm,
                  bottom=3cm]{geometry}
    \end{lstlisting}
    This is mostly done so that todonotes (from the \texttt{todonotes}-package)
    are more likely to have enough space. To that end the todonotes are set to
    have a \texttt{textwidth}-setting of 4cm: 
    \begin{lstlisting}[style=latex,gobble=6]
      \usepackage[textwidth=4cm,
                  textsize=tiny,
                  color=todored,
                  bordercolor=white]{todonotes}
    \end{lstlisting}

  \subsection{Listings}
    Listings (from the \texttt{listings}-package) have been customized to look
    appealing within the text and to support a variety of styles, such as Bash
    scripts, command line inputs, \TeX commands and languages like Java,
    Python, C, etc. 

  \subsection{Colors}
    \begin{tabulary}{\textwidth}{LLLL}
      \toprule 
      \textbf{color name} 
                  & \textbf{command} 
                                  & \textbf{hex definition} 
                                                      & \textbf{examples}     \\
      \midrule 
      \texttt{black} 
                  & \texttt{\textbackslash bk} 
                                  & \texttt{\#000000} & \colorbox{black!100}
                                                          {\wh back}
                                                          \bk text            \\
      \texttt{gray\_custom} 
                  &  
                                  & \texttt{\#999999} & \colorbox
                                                          {gray_custom!100}
                                                          {\wh back}
                                                          \color{gray_custom} 
                                                          text                \\
      \texttt{white} 
                  & \texttt{\textbackslash wh}
                                  & \texttt{\#ffffff} & \colorbox{white!100}
                                                          {\bk back}
                                                          \color{white} text  \\
      \texttt{blue\_custom} 
                  & \texttt{\textbackslash bl} 
                                  & \texttt{\#0066b5} & \colorbox
                                                          {blue_custom!100}
                                                          {\wh back}
                                                          \bl text            \\
      \texttt{red\_custom} 
                  & \texttt{\textbackslash re} 
                                  & \texttt{\#9e0007} & \colorbox
                                                          {red_custom!100}
                                                          {\wh back}
                                                          \re text            \\
      \texttt{green\_custom} 
                  & \texttt{\textbackslash gr} 
                                  & \texttt{\#007527} & \colorbox
                                                          {green_custom!100}
                                                          {\wh back}
                                                          \gr text            \\
      \bottomrule
    \end{tabulary}

  \subsection{Inline Math Explanations}
    To make simple non-invasive explanations in equations and such easy to
    write, two commands have been created: 
    \begin{itemize}
      \item \texttt{\textbackslash mexpl\{<arg>\}} \\ creates a simple one-line
            inline bracketed text block in grey
      \item \texttt{\textbackslash mexplbig\{<arg>\}} \\ creates a parbox of
            4cm width for inline bracketed \texttt{footnotesize} text block in
            grey, that has nice line-breaks and allows for full sentence
            explanations of steps 
    \end{itemize}
    An example of this can be found by the equations in subsection
    \ref{subsec:maths}. 


\section{Examples}
  \subsection{Code}
    What would a code listing look like? Here for example a listing of Java
    code: 
    \begin{lstlisting}[gobble=6]
      public class ExampleClass{
        private static int number = 69; 
        // example comment 
        public static void main(String args[]){
          System.out.println("Example Code for LaTeX Testing...");
        }
      }
    \end{lstlisting}

    Some Haskell: 
    \begin{lstlisting}[style=haskell,gobble=6]
      -- | calculates the factorial (n!) of a number n
      fac :: Integer -- ^ the value n for which to calculate the factorial 
          -> Integer -- ^ the factorial of n (n!)
      fac n 
        | n == 0          = 1 
        | otherwise       = n * ( fac (n - 1) )
    \end{lstlisting}

    And now some bash: 
    \begin{lstlisting}[style=bash,gobble=6]
      while true; do
        echo "example" 
        sleep 1
      done
    \end{lstlisting}
    As well as a terminal input: 
    \begin{lstlisting}[style=terminal,gobble=6]
      $ ffmpeg -i example.mp4 example.mp3
    \end{lstlisting}

  \subsection{Mathematics}\label{subsec:maths}
    To show off some equations and such, let's prove the irrationality of
    $\sqrt{2}$. 

  \subsubsection*{Proof by Contradiction}
    Suppose $\sqrt{2}$ were a rational number. 
    That would imply that there exist two integers $a,b\in\mathbb{Z}$ such that
    the fraction $\frac{a}{b} = \sqrt{2}$ is in lowest terms. \\
    Thus let $a,b$ be two such integers. 
    \begin{align*}
      &&
          \frac{a}{b}                     &=          \sqrt{2}
                                                                              \\
      \Rightarrow&&       
          \left(\frac{a}{b}\right)^2      &=          \left(\sqrt{2}\right)^2
          &&\mexpl{squaring both sides}
                                                                              \\
      \Rightarrow&&       
          \frac{a^2}{b^2}                 &=          2
          &&\mexplbig{
              squaring a fraction is squaring numerator and denominator
          }
                                                                              \\
      \Leftrightarrow&&       
          \frac{a^2}{b^2}(b^2)            &=          2(b^2)
          &&\mexpl{multipy with $b^2$}
                                                                              \\
      \Leftrightarrow&&       
          a^2                             &=          2b^2
          &&\mexpl{simplify}
    \end{align*}
    So $a^2$ is double the square of some integer $b$, therefore $a^2$ has to
    be an even number. 
    This means $a$ has to be an even number as well, since if it were odd, it's
    square would be odd as well.

    We also know of course that $a$ has to be greater than $0$, since
    $\frac{0}{b}^2 = 0 < 2$. 
    So $a^2$ has to be an even number's square, itself an even number and more
    imporantly still even if divided by 2. \\
    This means that since we also know $b^2 = \frac{a^2}{2}$ that $b^2$ has to
    be even, and again this implies that $b$ is even as well. 

    However this contradicts our supposition that $\frac{a}{b}$ is in lowest
    terms, since now we know they're both divisible by $2$. 

    Thus we've encountered a contradiction and know that our initial
    supposition has to be false. 

    Therefore $\nexists \,\, a,b\in\mathbb{Z} : \frac{a}{b} = \sqrt{2}$ meaning
    $\sqrt{2} \not\in \mathbb{Q}$. 
    \qed

  \subsubsection{common formatting questions}
    To have a brace under part of an equation use \verb|\underbrace|:  
    \begin{lstlisting}[style=LaTeX]
      $$n^2 = \underbrace{ n\cdot n\cdot...\cdot n }_{ n\text{-times} }$$
    \end{lstlisting}
      $$n^2 = \underbrace{ n\cdot n\cdot...\cdot n }_{ n\text{-times} }$$

    To simply write over an element of an equation, use \verb|\overset|. 
    If you'd like to write above an arrow you'd probably be better off using
    \verb|\xrightarrow| and similar commands as with these, the arrows extend
    with the text over and under them (\verb|\xrightarrow[under]{over}| takes
    an optional argument for under and an argument for over): 
    \begin{lstlisting}[style=LaTeX]
      $$\overset{ \delta }{\rightarrow}$$
      $$\overset{ a^2 = b^2 }{\Rightarrow}$$

      $$\xrightarrow{ \text{text above arrow} }$$
      $$\xRightarrow[ \text{text under arrow} ]{}$$
    \end{lstlisting}
      $$\overset{ \delta }{\rightarrow}$$
      $$\overset{ a^2 = b^2 }{\Rightarrow}$$

      $$\xrightarrow{ \text{text above arrow} }$$
      $$\xRightarrow[ \text{text under arrow} ]{}$$


  \subsection{Lorem Ipsum}
  Just to see what a couple of dense-text paragraphs would end up looking
  like, have a lorem ipsum (with a little todonote from the
  \texttt{todonotes}-package thrown in). 

\paragraph{Paragraph Title}
  Lorem ipsum dolor sit amet, consectetur adipiscing elit. Ut ut lorem a sem
  mattis commodo faucibus non diam. Duis feugiat, sapien a accumsan mattis,
  velit erat vulputate magna, nec scelerisque augue lectus sit amet quam. 
  Duis \textit{pellentesque} lorem sit amet odio lobortis ullamcorper. 
  Phasellus bibendum dapibus nulla, aliquet auctor tortor pellentesque non. 
  \textit{Nam nec metus varius}, semper turpis non, dapibus 
  justo. Cras lacinia faucibus massa eget vestibulum. Integer porta ante non
  risus finibus porttitor. Donec eget blandit nibh. Sed sodales lacus ut orci
  luctus volutpat. Vivamus ut dapibus mi. Nam lectus leo, fermentum sit amet
  pharetra in, porta id arcu.
  \todo{Example TODO Note with some content in it, simply to showcase that now
        these notes can contain some text without it being likely for you to
        run into spacing troubles, due to the margins}

  Sed tempor purus at ex eleifend varius. Aliquam tincidunt nibh eu est
  pellentesque aliquam. Pellentesque pharetra mi aliquam tincidunt vulputate.
  Suspendisse vulputate lobortis ultricies. Etiam rutrum eros bibendum sem
  elementum, at condimentum metus venenatis. Ut sed volutpat dui. Curabitur
  dapibus risus in congue tempor. Nullam at justo quam. Cras mi quam,
  faucibus eget erat vel, interdum finibus quam. Vestibulum quam nibh,
  bibendum quis ipsum at, maximus interdum purus. Duis id convallis mauris.
  Sed et suscipit augue.

\end{document}
